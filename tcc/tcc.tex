%!latex
\documentclass[12pt, a4paper, leqno, twoside]{book}

\usepackage{%
  fancyhdr,
  geometry,
  setspace,
  titlesec
}
\usepackage[brazil]{babel}

% Formatação do título dos capítulos
\titleformat{\chapter}[display]
{}%
{
   \Large\centering\normalsize%
   {\sc\chaptertitlename}\hspace{1ex}%
   {\selectfont\thechapter}%
}%
{3ex}% 
{\centering\bf\large\lsstyle\MakeUppercase}%
[]

% Espaçamento de letra para o cabeçalho dos capítulos
\usepackage[letterspace=203]{microtype}

% Formatação do título das se\c c\~oes
\titleformat{\section}{\centering\sc}{\thesection}{1em}{}[]

% Estilo dos cabeçalhos
\renewcommand{\chaptermark}[1]{\markboth{#1}{}}
\renewcommand{\sectionmark}[1]{\markright{\thesection\ #1}}
\renewcommand{\headrulewidth}{0ex}
\fancypagestyle{est}{%
  \def\fontface{\bf}
  \fancyhf{}
  \fancyhead[CE]{\fontface\sc\leftmark}
  \fancyhead[CO]{{\S}\fontface\sc\rightmark}
  \fancyhead[LE,RO]{\fontface\thepage}
}

% Numeração dos capítulos em algarismos romanos
\def\theHchapter{\arabic{chapter}\thechapter}
\def\thechapter{\Roman{chapter}}

% Remover páginas em branco
\let\cleardoublepage=\clearpage

% Geometria da página
\geometry{top=3cm, bottom=2cm, left=3cm, right=2cm}

% Definições gerais
\def\tpfc{\scshape}

\def\localData{%
  \parbox[c][0cm][c]{4cm}{
    \sc
    Jo\~ao Pessoa\hfill -\hfill PB\\
    \footnotesize
    \today
  }
}

\def\parteSuperior{
  \parbox[c][0cm][c]{12cm}{
    {\scshape\scriptsize%
      {INSTITUTO\hfill FEDERAL\hfill DE\hfill EDUCA\c C\~AO,\hfill CI\^ENCIA\hfill E\hfill TECNOLOGIA DA PARA\'IBA}\\
      {
        \fontsize{12}{12}\selectfont
        {\noindent CURSO\hfill T\'ECNICO\hfill SUBSEQUENTE\hfill EM\hfill ELETR\^ONICA}\\
        \large
        {\noindent TRABALHO\hfill DE\hfill CONCLUS\~AO\hfill DE\hfill CURSO}\\
      }
    }
  }
  \vfill 
  {\lsstyle\Large\scshape%
    PROJETO
  }
  \vskip .5cm
  \textit{por}
  \vskip .5cm
  {\tpfc Harllen Ara\'ujo de Sena}
  \par
  {\it e}
  \par
  {\tpfc Henrique Cirilo Costa}
  \vskip .5cm
  \textit{orientado pelo}
  \vskip .5cm
  {\tpfc Prof. Dr. C\'icero Alisson dos Santos}
  \vskip .5cm
}

\begin{document}

  % Remover a capa da contagem de páginas.
  \pagenumbering{gobble}

  \pagestyle{empty}
  \begin{center}
    \parteSuperior
    \vfill 
    \localData
  \end{center} 
  \newpage

  % Começar a contagem de páginas.
  \pagenumbering{arabic}

  \begin{center}
    \parteSuperior
    \vskip 2cm
    \hfill
    \parbox[c][0cm][c]{7.7cm}{
      Trabalho de conclus\~ao de curso apresentado ao IFPB.
    }
    \vfill 
    \localData
  \end{center} 

  % Espaçamento entre linhas.
  \setstretch{1.5}

  \tableofcontents

  % Remover paginação da tabela de conteúdos
  \addtocontents{toc}{\protect\thispagestyle{empty}}

  \newpage
  \singlespacing
  \pagestyle{est}

  \chapter*{Introdu\c c\~ao}

  \chapter{Preliminares}
  
  \section{Amplificadores Operacionais}
  
  \begin{thebibliography}{}
    \bibitem{Malvino}
    MALVINO, Albert Paul; BATES, David J. {\it Eletr\^onica\/}: Volume 1. 2. ed. S\~ao Paulo: McGraw-Hill Education, 1986.
    \bibitem{HH}
    HOROWITZ, Paul; HILL, Winfield; {\it The Art of Electronics.} 7. ed. New York: Cambridge University Press, 2016.
    \bibitem{TexIns}
    {\it LM555 Timer\/}: Texas Instruments, 2015. Dispon\'ivel em: https://www.ti.com/lit/ds\allowbreak/symlink/lm555.pdf. Acesso em: 26 novembro 2023.
  \end{thebibliography}
\end{document}
