%!latex
\documentclass[12pt, a4paper, leqno, twoside]{book}

\usepackage{%
  fancyhdr,
  geometry,
  setspace,
  titlesec
}
\usepackage[brazil]{babel}

% Formatação do título dos capítulos
\titleformat{\chapter}[display]
{}%
{
   \Large\centering\normalsize%
   {\sc\chaptertitlename}\hspace{1ex}%
   {\selectfont\thechapter}%
}%
{3ex}% 
{\centering\bf\large\lsstyle\MakeUppercase}%
[]

% Espaçamento de letra para o cabeçalho dos capítulos
\usepackage[letterspace=203]{microtype}

% Formatação do título das se\c c\~oes
\titleformat{\section}{\centering\sc}{\thesection}{1em}{}[]

% Estilo dos cabeçalhos
\renewcommand{\chaptermark}[1]{\markboth{#1}{}}
\renewcommand{\sectionmark}[1]{\markright{\thesection\ #1}}
\renewcommand{\headrulewidth}{0ex}
\fancypagestyle{est}{%
  \def\fontface{\bf}
  \fancyhf{}
  \fancyhead[CE]{\fontface\sc\leftmark}
  \fancyhead[CO]{{\S}\fontface\sc\rightmark}
  \fancyhead[LE,RO]{\fontface\thepage}
}

% Numeração dos capítulos em algarismos romanos
\def\theHchapter{\arabic{chapter}\thechapter}
\def\thechapter{\Roman{chapter}}

% Remover páginas em branco
\let\cleardoublepage=\clearpage

% Geometria da página
\geometry{top=3cm, bottom=2cm, left=3cm, right=2cm}

\renewcommand{\thefootnote}{\small\bfseries\arabic{footnote}}

% Definições gerais
\def\tpfc{\scshape}

\def\localData{%
  \parbox[c][0cm][c]{4cm}{
    \sc
    Jo\~ao Pessoa\hfill -\hfill PB\\
    \footnotesize
    \today
  }
}

\def\parteSuperior{
  \parbox[c][0cm][c]{12cm}{
    {\scshape\scriptsize%
      {INSTITUTO\hfill FEDERAL\hfill DE\hfill EDUCA\c C\~AO,\hfill CI\^ENCIA\hfill E\hfill TECNOLOGIA DA PARA\'IBA}\\
      {
        \fontsize{12}{12}\selectfont
        {\noindent CURSO\hfill T\'ECNICO\hfill SUBSEQUENTE\hfill EM\hfill ELETR\^ONICA}\\
        \large
        {\noindent TRABALHO\hfill DE\hfill CONCLUS\~AO\hfill DE\hfill CURSO}\\
      }
    }
  }
  \vfill 
  {\lsstyle\Large\scshape%
    PROJETO
  }
  \vskip .5cm
  \textit{por}
  \vskip .5cm
  {\tpfc Harllen Ara\'ujo de Sena}
  \par
  {\it e}
  \par
  {\tpfc Henrique Cirilo Costa}
  \vskip .5cm
  \textit{orientado pelo}
  \vskip .5cm
  {\tpfc Prof. Dr. C\'icero Alisson dos Santos}
  \vskip .5cm
}

\def\ohm{\,\Omega}
\def\opamp{OpAmp}

\begin{document}

  % Remover a capa da contagem de páginas.
  \pagenumbering{gobble}

  \pagestyle{empty}
  \begin{center}
    \parteSuperior
    \vfill 
    \localData
  \end{center} 
  \newpage

  % Começar a contagem de páginas.
  \pagenumbering{arabic}

  \begin{center}
    \parteSuperior
    \vskip 2cm
    \hfill
    \parbox[c][0cm][c]{7.7cm}{
      Trabalho de conclus\~ao de curso apresentado ao IFPB.
    }
    \vfill 
    \localData
  \end{center} 

  % Espaçamento entre linhas.
  \setstretch{1.5}

  \tableofcontents

  % Remover paginação da tabela de conteúdos
  \addtocontents{toc}{\protect\thispagestyle{empty}}

  \newpage
  \singlespacing
  \pagestyle{est}

  \chapter*{Introdu\c c\~ao}

  \chapter{Preliminares}
  
  \section{Amplificadores Operacionais}

  \chapter{Resumo do projeto}

  \section{Sobre o projeto} 
  A proposta do projeto \'e inovadora, no sentido de criar um amplificador de \'audio com baixa distor\c c\~ao\footnote{Embora intuitivo \'e necess\'ario precisar tecnicamente o que \'e distor\c c\~ao em \'audio.} e de baixo custo. Para este fim, usou-se v\'arios NE5532. Cada um consiste dum dual OpAmp (amplificador operacional dual), precisamente um {\it dual in-line package} (DIP) com dois amplificadores operacionais embutidos. O autor do projeto justifica a escolha deste CI devido \`a sua baixa distor\c c\~ao, \`a sua baixa imped\^ancia\footnote{Outro conceito a ser precisado.} de sa\'ida e \`a uma not\'avel performance de ru\'ido. A fim de suplantar o desafio t\'ecnico de se alimentar um alto-falante de $8\ohm$ com uma boa pot\^encia, faz-se o uso duma ponte ({\it bridge}). Conectam-se dois amplificadores em cascata (s\'erie), resultando num aumento de duas vezes a tens\~ao e, consequentemente quadruplicando a pot\^encia do sinal, sobrepujando o limiar de pot\^encia dum \'unico amplificador. Um outro fator prepoderante \'e o limite da corrente de sa\'ida de cada OpAmp, estipulado para evitar sua sobrecarga interna. Segundo o pr\'oprio autor do projeto o NE5532 acionar\'a uma carga de $500\ohm$\footnote{Creio que este par\^ametro \'e dependente do fabricante.} at\'e o limiar da tens\~ao de sa\'ida do \opamp, embora seja recomend\'avel usar cargas mais ``leves'', isto \'e, cargas com resist\^encias maiores. O projeto foi dimensionado para alimentar um alto-falante de $8\ohm$, caso $4\ohm$ seja requerido, ser\~ao necess\'arios duas vezes mais \opamp{s}, para fornecer o dobro de corrente demandada pela carga de $4\ohm$ e, o mesmo se aplica ao modo de opera\c c\~ao {\it bridged\/}\footnote{Neste modo, a carga, a saber, o alto-falante, receber\'a duas tens\~oes invertidas em fase, isto por sua vez resultar\'a na duplica\c c\~ao da tens\~ao de sa\'ida e na quadruplica\c c\~ao da pot\^encia.}. O sistema foi desenvolvido para que os modos {\it single-ended\/}\footnote{A carga ser\'a conectada ao GND e a tens\~ao de sa\'ida.} e {\it bridged}. Ademais, devido \`a sua modularidade \'e poss\'ivel construir um amplificador est\'ereo\footnote{Precipuamente, a configura\c c\~ao est\'ereo \'e constituida de dois canais um esquerdo ({\bf L}{\it eft}) e um direito ({\bf R}{\it ight}).} apenas com tr\^es PCIs. \'E sabido que inerentemente os \opamp{s} possuem prote\c c\~ao contra sobrecarga. N\~ao obstante, rel\'es de sa\'ida s\~ao usados para evitar o {\it on-off muting} causador dos efeitos indesejados ao se ligar um sistema de \'audio, e.g. os estalos ({\it pops}), e para evitar falhas DC, i.e., evitar que o sistema forneça DC ao alto-falante evitando assim, sua sobrecarga. 

  \section{Um tour pelos est\'agios}
  
  \begin{thebibliography}{}
    \bibitem{Malvino}
    MALVINO, Albert Paul; BATES, David J. {\it Eletr\^onica\/}: Volume 1. 2. ed. S\~ao Paulo: McGraw-Hill Education, 1986.
    \bibitem{HH}
    HOROWITZ, Paul; HILL, Winfield; {\it The Art of Electronics.} 7. ed. New York: Cambridge University Press, 2016.
%    \bibitem{TexIns}
%    {\it LM555 Timer\/}: Texas Instruments, 2015. Dispon\'ivel em: https://www.ti.com/lit/ds\allowbreak/symlink/lm555.pdf. Acesso em: 26 novembro 2023.
  \end{thebibliography}
\end{document}
