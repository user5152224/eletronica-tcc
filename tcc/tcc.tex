%!latex
\documentclass[12pt, a4paper, leqno, twoside]{book}

\usepackage{%
  fancyhdr,
  geometry,
  setspace,
  titlesec
}
\usepackage[brazil]{babel}

% Formatação do título dos capítulos
\titleformat{\chapter}[display]
{}%
{
   \Large\centering\normalsize%
   {\sc\chaptertitlename}\hspace{1ex}%
   {\selectfont\thechapter}%
}%
{3ex}% 
{\centering\bf\large\lsstyle\MakeUppercase}%
[]

% Espaçamento de letra para o cabeçalho dos capítulos
\usepackage[letterspace=203]{microtype}

% Formatação do título das se\c c\~oes e subseções
\titleformat{\section}{\centering\large\scshape}{\thesection}{1em}{}[]
\titleformat{\subsection}{\centering\scshape}{\thesubsection}{1em}{}[]

% Estilo dos cabeçalhos
\renewcommand{\chaptermark}[1]{\markboth{#1}{}}
\renewcommand{\sectionmark}[1]{\markright{\thesection\ #1}}
\renewcommand{\headrulewidth}{0ex}
\fancypagestyle{est}{%
  \def\fontface{\bf}
  \fancyhf{}
  \fancyhead[CE]{\fontface\sc\leftmark}
  \fancyhead[CO]{{\S}\fontface\sc\rightmark}
  \fancyhead[LE,RO]{\fontface\thepage}
}

% Numeração dos capítulos em algarismos romanos
\def\theHchapter{\arabic{chapter}\thechapter}
\def\thechapter{\Roman{chapter}}

% Remover páginas em branco
\let\cleardoublepage=\clearpage

% Geometria da página
\geometry{top=3cm, bottom=2cm, left=3cm, right=2cm}

% Modificar o estilo da numeração das notas de rodapé
\renewcommand{\thefootnote}{\bfseries\arabic{footnote}}

% Definições gerais
\def\tpfc{\scshape}

\def\localData{%
  \parbox[c][0cm][c]{4cm}{
    \sc
    Jo\~ao Pessoa\hfill -\hfill PB\\
    \footnotesize
    \today
  }
}

\def\parteSuperior{
  \parbox[c][0cm][c]{12cm}{
    {\scshape\scriptsize%
      {INSTITUTO\hfill FEDERAL\hfill DE\hfill EDUCA\c C\~AO,\hfill CI\^ENCIA\hfill E\hfill TECNOLOGIA DA PARA\'IBA}\\
      {
        \fontsize{12}{12}\selectfont
        {\noindent CURSO\hfill T\'ECNICO\hfill SUBSEQUENTE\hfill EM\hfill ELETR\^ONICA}\\
        \large
        {\noindent TRABALHO\hfill DE\hfill CONCLUS\~AO\hfill DE\hfill CURSO}\\
      }
    }
  }
  \vfill 
  {\lsstyle\Large\scshape%
    PROJETO }
  \vskip .5cm
  \textit{por}
  \vskip .5cm
  {\tpfc Harllen Ara\'ujo de Sena}
  \vskip .15cm
  {\it e}
  \vskip .15cm
  {\tpfc Henrique Cirilo Costa}
  \vskip .5cm
  \textit{orientado pelo}
  \vskip .5cm
  {\tpfc Prof. Dr. C\'icero Alisson dos Santos}
  \vskip .5cm
}

\def\ohm{\,\Omega}
\def\ampop{AmpOp}
\def\db{{\rm dB}}
\def\[{\vskip .5cm\begin{equation}}
\def\]{\end{equation}\vskip .5cm\noindent}

\begin{document}

  % Remover a capa da contagem de páginas.
  \pagenumbering{gobble}

  \pagestyle{empty}
  \begin{center}
    \parteSuperior
    \vfill 
    \localData
  \end{center} 
  \newpage

  % Começar a contagem de páginas.
  \pagenumbering{arabic}

  \begin{center}
    \parteSuperior
    \vskip 2cm
    \hfill
    \parbox[c][0cm][c]{7.7cm}{
      Trabalho de conclus\~ao de curso apresentado ao IFPB.
    }
    \vfill 
    \localData
  \end{center} 

  % Espaçamento entre linhas.
  \setstretch{1.5}

  \tableofcontents

  % Remover paginação da tabela de conteúdos
  \addtocontents{toc}{\protect\thispagestyle{empty}}

  \newpage
  \singlespacing
  \pagestyle{est}

  \chapter*{Introdu\c c\~ao}

  \chapter{Preliminares}
  
  \section{Amplificadores Operacionais}

  \chapter{Resumo do projeto}

  \section{Sobre o projeto} 
  A proposta do projeto \'e inovadora no sentido que ela prop\~oe criar um amplificador de \'audio com baixa distor\c c\~ao\footnote{Embora intuitivo \'e necess\'ario precisar tecnicamente o que \'e distor\c c\~ao em \'audio.} e de baixo custo, usando uma combina\c c\~ao de v\'arios CIs NE5532. Cada um consiste dum amplificador operacional (\ampop) dual, precisamente, um {\it Dual In-Line Package} (DIP) com dois amplificadores operacionais embutidos. O autor do projeto justifica a escolha deste CI devido \`a sua baixa distor\c c\~ao, \`a sua baixa imped\^ancia\footnote{Outro conceito a ser precisado.} de sa\'ida e \`a uma not\'avel performance de ru\'ido. 

  A fim de suplantar o desafio t\'ecnico de alimentar um alto-falante de $8\ohm$ com uma pot\^encia aceit\'avel, faz-se o uso duma ponte ({\it Bridge}). Conectam-se dois amplificadores em cascata (s\'erie), resultando num aumento de duas vezes a tens\~ao e, consequentemente quadruplicando a pot\^encia do sinal, sobrepujando o limiar de pot\^encia dum \'unico amplificador. 

  Um outro fator preponderante \'e o limite da corrente de sa\'ida de cada \ampop, que por sua vez \'e estipulado para evitar sua sobrecarga. Segundo o pr\'oprio autor do projeto, o NE5532 consegue acionar uma carga de $500\ohm$\footnote{Creio que este par\^ametro \'e dependente do fabricante.} at\'e o limiar da tens\~ao de sa\'ida do \ampop. Entretanto, \'e recomend\'avel usar cargas mais ``leves'', isto \'e, cargas com resist\^encias maiores. 

  O projeto foi dimensionado para alimentar um alto-falante de $8\ohm$, caso o de $4\ohm$ seja requerido, ser\~ao necess\'arios duas vezes mais \ampop{s}, para fornecer o dobro de corrente demandada pela carga de $4\ohm$ e, o mesmo se aplica ao modo de opera\c c\~ao {\it Bridged\/}\footnote{Neste modo, a carga, a saber, o alto-falante, receber\'a duas tens\~oes invertidas em fase, isto por sua vez resultar\'a na duplica\c c\~ao da tens\~ao de sa\'ida e {\it a fortiori} na quadruplica\c c\~ao da pot\^encia.}. 

  O sistema foi desenvolvido de maneira modular, para abarcar os modos {\it Single-Ended\/}\footnote{A carga ser\'a conectada ao GND e a tens\~ao de sa\'ida.} e {\it Bridged}. Ademais, devido \`a sua modularidade \'e poss\'ivel construir um amplificador est\'ereo\footnote{Precipuamente, a configura\c c\~ao est\'ereo \'e constitu\'ida de dois canais um esquerdo ({\bf L}{\it eft}) e um direito ({\bf R}{\it ight}).} com apenas tr\^es PCIs.

  \'E sabido que inerentemente os \ampop{s} possuem prote\c c\~ao contra sobrecarga. N\~ao obstante, rel\'es de sa\'ida s\~ao usados para evitar o {\it On-Off Muting} causador dos efeitos indesejados ao se ligar um sistema de \'audio, e.g., os estalos ({\it pops}); e para evitar falhas DC, i.e., evitar que o sistema forneça um sinal DC intermitente ao alto-falante, precavendo assim, a degenera\c c\~ao da bobina por aquecimento\footnote{Efeito Joule.} e do sistema de suspens\~ao do cone por deforma\c c\~ao cont\'inua. 
  \section{Um tour pelos designs}
  \subsection{A entrada desbalanceada}

  Este est\'agio consiste de um filtro RF, neste caso um filtro passa-baixas, pois a tens\~ao de sa\'ida \'e 
  \begin{equation}
    \bigg|{R_2\|Z_{C_1}\over R_1+R_2\|Z_{C_1}}\bigg|\cdot V_{\rm in}=\bigg|{R_2\over\omega C_1R_1R_2-j(R_1+R_2)}\bigg|\cdot V_{\rm in}
  \end{equation}
  em que $V_{\rm in}$ \'e a tens\~ao de entrada. Esta entrada \'e chamada de desbalanceada, pois est\'a mais sucet\'ivel \`a interfer\^encia eletromagn\'etica {\it Radio Frequency} (RF), por exemplo proveniente do uso cabos longos. Ela pode ser conectada diretamente ao est\'agio de ganho---tratado nas pr\'oximas subse\c c\~oes---atrav\'es de um jumper em JP1. 

  \subsection{A entrada balanceada}
  Um est\'agio convencional \'e constru\'ido com quatro resistores de $10\,{\rm k}\Omega$ e um \'unico \ampop\ 5532, ele tem uma performace de ru\'ido pior que uma entrada desbalanceada simples. Al\'em disso, o ru\'ido \'e ainda pior que a maioria dos amplificadores de pot\^encia. O amplificador balanceado soluciona este problema parcialmente. Trata-se dum est\'agio amplificador balanceado duplo ({\it Dual Balanced Stage Amplifier}) compreendendo aos \ampop{s}\ IC5A e IC5B, que cancela parcialmente ru\'ido n\~ao correlacionado---ru\'ido aleat\'orio sem rela\c c\~ao aos dois \ampop{s}---dando uma redu\c c\~ao de ru\'ido de $3\,\db$, melhorando assim o CMRR\footnote{Definir este conceito!}. Ele tamb\'em usa resistores de valores muito menores, a saber, $802\,\ohm$ se comparado com os usados ordinariamente, {\it viz.} $10\,{\rm k}\Omega$, engendrando assim num ru\'ido Jonhson\footnote{\'E sabido que um resistor antigo apenas estando sobre uma mesa, gera uma tens\~ao de ru\'ido atrav\'es de seus terminais conhecido como ru\'ido Johnson. Ele tem um {\it spectrum} de frequ\^encia achatado dentro de uma banda de frequ\^encias. Ru\'idos com {\it spectra} achatados s\~ao comumente classificados como ru\'ido branco ({\it White Noise}\/). O ru\'ido Jonhson (Nyquist), \'e um ru\'ido rand\^omico inerente aos condutores el\'etricos em equil\'ibrio t\'ermico, associado \`a agita\c c\~ao t\'ermica dos portadores majorit\'arios de carga (usualmente os el\'etrons) e indiferente \`a diferen\c ca de potencial no condutor. A tens\~ao de ru\'ido de circuito aberto gerado por uma resist\^encia $R$ em $\ohm$ \`a temperatura $T$ em Kelvin $[T(t)\,{\rm K}=(t+273)\,^\circ{\rm C}]$ \'e na realidade dada pela express\~ao:
    $$
      v_{J}({\rm rms})=\sqrt{4kBRT}\quad{\rm V(rms)},
    $$
    em que $k$ \'e a constante de Boltzmann e $B$ \'e o comprimento da banda em Hz.
  } ({\it Johnson Noise}) menor. Isso s\'o \'e poss\'ivel porque o amplificador \'e controlado pelos buffers, que permitem que a imped\^ancia de entrada sejam mais altas que o usual, evitando a sobrecarga dos equipamentos externos, melhorando ainda mais o CMRR. O ru\'ido de sa\'ida \'e de menos de $-112\,\db{\rm u}$, uma melhora de $8\,\db$ relativo \`a tecnologia convencional.
  \subsection{O est\'agio de ganho}
  De acordo com o autor do projeto, o amplificador principal, assim como o est\'agio balanceado, possui um design inovador, que obtem uma distor\c c\~ao muito baixa distribuindo o ganho requerido sobre tr\^es est\'agios. Sabe-se que $22.7\,\db$ poderia ser facilmente obtido com um \'unico \ampop. N\~ao obstante, os NE5532s n\~ao est\~ao completamente livres de distor\c c\~ao, e o THD ({\it Total Harmonic Distortion}\/) seria significativo.

  O primeiro est\'agio (IC1A,1C1B) d\'a um ganho de $10.7\,\db$[\footnote{Eu calculei $20\cdot\log_{10}(6220/910)\approx16.7\,\db$.}; as duas sa\'idas s\~ao combinadas por $R_8$ e $R_9$ engendrando uma vantagem de $3\,\db$\footnote{N\~ao entendi o porqu\^e.}, como no amplicador balanceado. O segundo est\'agio duplica a tens\~ao, pois
  \[
    V_{\rm out}=\biggl(1+{R_{11}\over R_{10}}\biggr)\cdot V_{\rm in}=\biggl(1+{2{\rm k}2\over 2{\rm k}2}\biggr)\cdot V_{\rm in}=2\cdot V_{\rm in},
  \]
  donde o ganho em $\db$ \'e
  \[
    20\cdot\log_{10}\biggr({V_{\rm out}\over V_{\rm in}}\biggr)=20\cdot\log_{10}2\approx 6\,\db.
  \]
  O ganho \'e menor para maximizar a retroalimenta\c c\~ao negativa ({\it Negative Feedback}\/), porque agora o n\'ivel de tens\~ao \'e mais alto. 

  IC2B \'e um seguidor de tens\~ao, logo, seu ganho \'e unit\'ario. Sua fun\c c\~ao \'e prevenir a imped\^ancia de entrada de $1\,{\rm k}\Omega$ do est\'agio de ganho final IC3B, de carregar a sa\'ida de IC2A, causando distor\c c\~ao. IC2B \'e menos vulner\'avel \`a carga, porque ela tem uma retroalimenta\c c\~ao negativa m\'axima. 

  Em IC3B, no pino 6, em virtude de \ampop{s} possuirem imped\^ancia de entrada muito alta, a corrente nas entradas do \ampop\ s\~ao neglig\'iveis. Portanto, idealmente, segue da lei de Kirchhoff a igualdade
  \[
    {V_{\rm in}\over R_{12}}+{V_{\rm out}\over R_{13}+R_{14}}=0,
  \]
  ou seja,
  \[
    {V_{\rm out}\over V_{\rm in}}=-{R_{13}+R_{14}\over R_{12}}={2047\over 1000}\approx 2,
  \]
  logo o ganho em \db\ \'e aproximadamente
  \[
    20\cdot\log_{10}2\approx6\,\db. 
  \]
  Ele \'e usado no modo {\it shunt-feedback} para evitar a distor\c c\~ao de modo comum, que resultariam de sinais de altos n\'iveis. Ele tem uma sa\'ida do tipo `imped\^ancia-zero', com retroalimentac\~ao HF ({\it High Frequency}\/) via $C_8$ e LF ({\it Low Frequency}\/) via $R_{13}$. Desta maneira, o {\it crosstalk}\footnote{O que \'e isso?} \'e mantido no m\'inimo, enquanto mant\'em estabilidade com carga capacitiva\footnote{Isto est\'a correto?}. 

  A sa\'ida em $K_3$ est\'a revertida em fase e pode ser usada para {\it Bridging}\footnote{???}.

  IC3A \'e um est\'agio inversor de ganho unit\'ario que corrige a fase do sinal. A sa\'ida tamb\'em \'e do tipo `imped\^ancia-zero'.

  \subsection{O amplificador de pot\^encia}

  O amplificador de pot\^encia consiste de trinta e dois 5532 dual \ampop, logo, no total s\~ao 64 se\c c\~oes de \ampop{s}, atuando como seguidores de tens\~ao, com suas sa\'idas reunidas por resistores de $1\,\ohm$. Estes resistores combinados est\~ao fora dos la\c cos de retroalimenta\c c\~ao dos 5532s.

  (Incluir o {\it Damping Factor}).

  A fila de \ampop{s} s\~ao unidos pelos jumpers $K_5$--$K_{12}$.

  Existe uma {\it Output Choke} $L_1$ para estabilidade em cargas capacitivas, e {\it Catching Diodes} $D_1$--$D_2$ para previnir danos devidos aos transientes\footnote{Geralmente s\~ao picos ou impulsos el\'etricos indesejados de curta dura\c c\~ao, causador de estalos nos alto-falantes, e.g., no momento em que se liga e desliga o sistema de \'audio.} de tens\~ao ao limitar a corrente em cargas reativas. 
  \begin{thebibliography}{}
    \bibitem{HH}
    HOROWITZ, Paul; HILL, Winfield; {\it The Art of Electronics.} 7. ed. New York: Cambridge University Press, 2016.
    \bibitem{Malvino}
    MALVINO, Albert Paul. {\it Eletr\^onica\/}: Volume 1. 1ª ed. S\~ao Paulo: McGraw-Hill Education, 1987.
    \bibitem{SelfElektor}
    SELF, Douglas. {\it The 5532 OpAmplifier. Part 1: design philosophy and schematics.} Elektor, p. 14--17, out. 2010.
    \bibitem{SelfGiesbertsElektor}
    SELF, Douglas; GIESBERTS, Ton. {\it The 5532 OpAmplifier. Part 2: construction, bridged operation and test results.} Elektor, p. 24--27, nov. 2010.
  \end{thebibliography}
\end{document}
